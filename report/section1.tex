\section{Lexical Analysis}

The lexical analysis phase makes use of three components. A
\texttt{DFSA} class, a generic table-driven lexer, and a lexer
builder.

\subsection{The \texttt{DFSA}}

The \texttt{DFSA} class is an almost-faithful implementation of
the formal concept of a \texttt{DFSA}. \listref{dfsadecl},
outlines the behaviour of the \texttt{DFSA} class. Additionally,
it contains a number of helper functions which facilitate
getting the initial state and checking whether a state or a
transition \emph{category} is valid. The
\texttt{getInitialState()} helper is present, becuase after
consturction the \texttt{DFSA} does not guarantee, that the
initial state used by the user will be the same.

\lstinputlisting[
firstline=14,
lastline=44,
caption={\texttt{DFSA} class declaration (lexer/DFSA.hpp).},
label=lst:dfsadecl
]{lexer/DFSA.hpp}

The only significant difference from a formal \texttt{DFSA} is
the \texttt{getTransition()} method. In fact, it accepts a
vector of transition \emph{categories} instead of a single
\emph{category}.

This is because a symbol e.g. `\texttt{a}' or `\texttt{9}',
might be in multiple \emph{categories}. For instance
`\texttt{a}' is considered to be both a letter ($L$) and a
hexadecimal ($H$).

\subsubsection{Micro-Syntax \& \emph{Categories}}

The DFSA for accepting \texttt{PArL}'s micro-syntax is built as
follows.

Let $\mathfrak{U}$ be the set of all possible characters under
the system encoding (e.g. UTF-8).

The will use the following \emph{categories}:

\begin{itemize}
    \item $L \coloneq \{
        \texttt{A},\ldots,\texttt{Z},\texttt{a},\ldots,\texttt{z}\}$
    \item $D \coloneq
        \{\texttt{0},\ldots,\texttt{9}\}$
    \item $H \coloneq
        \{\texttt{A},\ldots,\texttt{F},\texttt{a},\ldots,\texttt{f}\}
        \cup D$
    \item $S \coloneq \{\alpha \in \mathfrak{U}
        \colon \alpha\ \text{is
    whitespace}\}\setminus\{\texttt{LF}\}$
\end{itemize}

\begin{noteline}
\texttt{LF} refers to line-feed or as it is more commonly known
`\texttt{\textbackslash n}' i.e. new-line.
\end{noteline}

\pagebreak

Together these \emph{categories} form our alphabet
$\Sigma$:

$$\Sigma \coloneq L \cup D \cup S \cup
\{\texttt{.},\texttt{\#},\texttt{\_},\texttt{(},\texttt{)},\texttt{[},\texttt{]},\texttt{\{},\texttt{\}},\texttt{*},\texttt{/},\texttt{+},\texttt{-},\texttt{<},\texttt{>},\texttt{=},\texttt{!},\texttt{,},\texttt{:},\texttt{;},\texttt{LF}\}$$

For improved readability the \texttt{DFSA}, which is usually a
single drawing, has been split across multiple drawings. Hence,
in each drawing  initial state ($0$) refers to the \textbf{same}
initial state (a \texttt{DFSA} has one and only one initial
state).

Additionally, each final state is annotated with
the token type it should produce.

\tikzset{
node distance=2cm, % specifies the minimum distance between two nodes. Change if necessary.
every state/.style={thick, fill=gray!10}, % sets the properties for each ’state’ node
initial text=$\text{start}$, % sets the text that appears on the start arrow
}

\label{sss:dfsadrawings}

\begin{figure}[H]
\centering
\begin{tikzpicture}
\node[state,initial] (q1) {$0$};
\coordinate[right of=q1] (tq1) {};
\node[state, right of=tq1, accepting] (q2) {};
\node [below right = 0.3cm and -1cm of q2]
    {\texttt{Token::Type::WHITESPACE}};
\draw
    (q1) edge[above, ->] node{$S\cup\{\texttt{LF}\}$} (q2)
    (q2) edge[loop above, ->] node{$S\cup\{\texttt{LF}\}$} (q2);
\end{tikzpicture}
\caption{States \& transitions for recognising
whitespace.}
\end{figure}

\begin{figure}[H]
\centering
\begin{tikzpicture}
\node[state,initial] (q1) {$0$};
\node[state, right of=q1, accepting] (q2) {};
\node [below right = 0.3cm and -1cm of q2]
    {\texttt{Token::Type::IDENTIFIER}};
\draw
    (q1) edge[above, ->] node{$L$} (q2)
    (q2) edge[loop above, ->] node{$L \cup D \cup \{\texttt{\_}\}$} (q2);
\end{tikzpicture}
\caption{States \& transitions for recognising
identifiers/keywords.}
\end{figure}

\begin{figure}[H]
\centering
\begin{tikzpicture}
\node[state,initial] (q1) {$0$};
\node[state, right of=q1] (q2) {};
\node[state, right of=q2] (q3) {};
\node[state, right of=q3, accepting] (q4) {};
\node [below right = 0.3cm and -1cm of q4]
    {\texttt{Token::Type::BUILTIN}};

\draw
    (q1) edge[above, ->] node{\texttt{\_}} (q2)
    (q2) edge[above, ->] node{\texttt{\_}} (q3)
    (q3) edge[above, ->] node{$L$} (q4)
    (q4) edge[loop above, ->] node{$L \cup D \cup \{\texttt{\_}\}$} (q4);
\end{tikzpicture}
\caption{States \& transitions for recognising
builtins.}
\end{figure}


\begin{figure}[H]
\centering
\begin{tikzpicture}
\node[state,initial] (q1) {$0$};
\node[state, right of=q1, accepting] (q2) {};
\node[state, right of=q2, accepting] (q3) {};
\node [above left = 0.3cm and -1cm of q2]
    {\texttt{Token::Type::MINUS}};
\node [below right = 0.3cm and -1cm of q3]
    {\texttt{Token::Type::ARROW}};
\draw
    (q1) edge[above, ->] node{\texttt{-}} (q2)
    (q2) edge[above, ->] node{\texttt{>}} (q3);
\end{tikzpicture}
\caption{States \& transitions for recognising `minus' and
`arrow' (\texttt{->}).}
\end{figure}

\begin{figure}[H]
\centering
\begin{tikzpicture}
\node[state,initial] (q1) {$0$};
\node[state, right of=q1, accepting] (q2) {};
\node[state, right of=q2] (q3) {};
\node[state, right of=q3, accepting] (q4) {};
\node [below left = 0.3cm and -1cm of q2]
    {\texttt{Token::Type::INTEGER}};
\node [below right = 0.3cm and -1cm of q4]
    {\texttt{Token::Type::FLOAT}};
\draw
    (q1) edge[above, ->] node{$D$} (q2)
    (q2) edge[loop above, ->] node{$D$} (q2)
    (q2) edge[above, ->] node{\texttt{.}} (q3)
    (q3) edge[above, ->] node{$D$} (q4)
    (q4) edge[loop above, ->] node{$D$} (q4)
    ;
\end{tikzpicture}
\caption{States \& transitions for recognising
integers and floats.}
\end{figure}

\begin{figure}[H]
\centering
\begin{tikzpicture}
\node[state,initial] (q1) {$0$};
\node[state, right of=q1] (q2) {};
\node[state, right of=q2] (q3) {};
\node[state, right of=q3] (q4) {};

\coordinate[below of=q1] (bq1) {};
\coordinate[below of=q4] (bq4) {};

\node[state, below of=bq1] (q5) {};
\node[state, right of=q5] (q6) {};
\node[state, right of=q6] (q7) {};
\node[state, right of=q7, accepting] (q8) {};
\node [below right = 0.3cm and -1cm of q8]
    {\texttt{Token::Type::COLOR}};
\draw
    (q1) edge[above, ->] node{\texttt{\#}} (q2)
    (q2) edge[above, ->] node{$H$} (q3)
    (q3) edge[above, ->] node{$H$} (q4)
    (q4) -- (bq4)
    (bq4) edge[above] node{$H$} (bq1)
    (bq1) edge[->] (q5)
    (q5) edge[above, ->] node{$H$} (q6)
    (q6) edge[above, ->] node{$H$} (q7)
    (q7) edge[above, ->] node{$H$} (q8)
    ;
\end{tikzpicture}
\caption{States \& transitions for recognising
colours.}
\end{figure}

\begin{figure}[H]
\centering
\begin{tikzpicture}
\node[state,initial] (q1) {$0$};
\node[state, right of=q1,accepting] (q2) {};

\coordinate[below of=q2] (bq2) {};

\node[state, right of=q2] (q4) {};
\node[state, right of=q4] (q5) {};
\node[state, right of=q5,accepting] (q6) {};

\node[state, below of=q5, accepting] (q3) {};

\node [above left = 0.3cm and -1cm of q2]
    {\texttt{Token::Type::SLASH}};

\node [above right = 0.3cm and -1cm of q6]
    {\texttt{Token::Type::COMMENT}};

\node [below left = 0.3cm and -1cm of q3]
    {\texttt{Token::Type::COMMENT}};

\draw
    (q1) edge[above, ->] node{\texttt{/}} (q2)
    (q2) -- (bq2)
    (bq2) edge[above, ->] node{\texttt{/}} (q3)
    (q3) edge[loop right, ->] node{$\Sigma \setminus \{\texttt{LF}\}$} (q3)
    (q2) edge[above, ->] node{\texttt{*}} (q4)
    (q4) edge[loop above, ->] node{$\Sigma \setminus \{\texttt{*}\}$} (q4)
    (q4) edge[above, bend left, ->] node{\texttt{*}} (q5)
    (q5) edge[below, bend left, ->] node{$\Sigma \setminus \{\texttt{/}\}$} (q4)
    (q5) edge[above, ->] node{\texttt{/}} (q6)
    ;
\end{tikzpicture}
\caption{States \& transitions for recognising `division' and
comments.}
\label{fig:comments}
\end{figure}

\begin{figure}[H]
\centering
\begin{tikzpicture}
\node[state,initial] (q1) {$0$};
\node[state, right of=q1, accepting] (q2) {};
\node[state, right of=q2, accepting] (q3) {};
\node [above left = 0.3cm and -1cm of q2]
    {\texttt{Token::Type::EQUAL}};
\node [below right = 0.3cm and -1cm of q3]
    {\texttt{Token::Type::EQUAL\_EQUAL}};
\draw
    (q1) edge[above, ->] node{\texttt{=}} (q2)
    (q2) edge[above, ->] node{\texttt{=}} (q3);
\end{tikzpicture}
\caption{States \& transitions for recognising `assign'
and `is equal to'.}
\end{figure}

\begin{figure}[H]
\centering
\begin{tikzpicture}
\node[state,initial] (q1) {$0$};
\node[state, right of=q1] (q2) {};
\node[state, right of=q2, accepting] (q3) {};
\node [below right = 0.3cm and -1cm of q3]
    {\texttt{Token::Type::BANG\_EQUAL}};
\draw
    (q1) edge[above, ->] node{\texttt{!}} (q2)
    (q2) edge[above, ->] node{\texttt{=}} (q3);
\end{tikzpicture}
\caption{States \& transitions for recognising `not equal to'.}
\end{figure}


\begin{figure}[H]
\centering
\begin{tikzpicture}
\node[state,initial] (q1) {$0$};
\node[state, right of=q1, accepting] (q2) {};
\node[state, right of=q2, accepting] (q3) {};
\node [above left = 0.3cm and -1cm of q2]
    {\texttt{Token::Type::LESS}};
\node [below right = 0.3cm and -1cm of q3]
    {\texttt{Token::Type::LESS\_EQUAL}};
\draw
    (q1) edge[above, ->] node{\texttt{<}} (q2)
    (q2) edge[above, ->] node{\texttt{=}} (q3);
\end{tikzpicture}
\caption{States \& transitions for recognising `less than' and
`less than or equal to'.}
\end{figure}

\begin{figure}[H]
\centering
\begin{tikzpicture}
\node[state,initial] (q1) {$0$};
\node[state, right of=q1, accepting] (q2) {};
\node[state, right of=q2, accepting] (q3) {};
\node [above left = 0.3cm and -1cm of q2]
    {\texttt{Token::Type::GREATER}};
\node [below right = 0.3cm and -1cm of q3]
    {\texttt{Token::Type::GREATER\_EQUAL}};
\draw
    (q1) edge[above, ->] node{\texttt{>}} (q2)
    (q2) edge[above, ->] node{\texttt{=}} (q3);
\end{tikzpicture}
\caption{States \& transitions for recognising `greater than'
and `greater than or equal to'.}
\end{figure}


\begin{figure}[H]
\centering
\begin{tikzpicture}
\node[state,initial] (q1) {$0$};

\coordinate[below of=q1] (b1) {};
\coordinate[below of=b1] (b2) {};
\coordinate[below of=b2] (b3) {};
\coordinate[below of=b3] (b4) {};
\coordinate[below of=b4] (b5) {};
\coordinate[below of=b5] (b6) {};

\node[state, right of=b1, accepting] (q2) {};
\node[state, below of=q2, accepting] (q3) {};
\node[state, below of=q3, accepting] (q4) {};
\node[state, below of=q4, accepting] (q5) {};
\node[state, below of=q5, accepting] (q6) {};

\node[state, left of=b1, accepting] (q7) {};
\node[state, below of=q7, accepting] (q8) {};
\node[state, below of=q8, accepting] (q9) {};
\node[state, below of=q9, accepting] (q10) {};
\node[state, below of=q10, accepting] (q11) {};
\node[state, below of=q11, accepting] (q12) {};

\node [below right = 0.3cm and -1cm of q2]
    {\texttt{Token::Type::STAR}};
\node [below right = 0.3cm and -1cm of q3]
    {\texttt{Token::Type::LEFT\_PAREN}};
\node [below right = 0.3cm and -1cm of q4]
    {\texttt{Token::Type::RIGHT\_PAREN}};
\node [below right = 0.3cm and -1cm of q5]
    {\texttt{Token::Type::LEFT\_BRACK}};
\node [below right = 0.3cm and -1cm of q6]
    {\texttt{Token::Type::RIGHT\_BRACK}};

\node [below left = 0.3cm and -1cm of q7]
    {\texttt{Token::Type::LEFT\_BRACE}};
\node [below left = 0.3cm and -1cm of q8]
    {\texttt{Token::Type::RIGHT\_BRACE}};
\node [below left = 0.3cm and -1cm of q9]
    {\texttt{Token::Type::COLON}};
\node [below left = 0.3cm and -1cm of q10]
    {\texttt{Token::Type::COMMA}};
\node [below left = 0.3cm and -1cm of q11]
    {\texttt{Token::Type::SEMICOLON}};
\node [below left = 0.3cm and -1cm of q12]
    {\texttt{Token::Type::PLUS}};

\draw
    (q1) -- (b1)
    (b1) -- (b2)
    (b2) -- (b3)
    (b3) -- (b4)
    (b4) -- (b5)
    (b5) -- (b6)
    ;

\draw
    (b1)  edge[above, ->] node{\texttt{*}} (q2)
    (b2) edge[above, ->] node{\texttt{(}} (q3)
    (b3) edge[above, ->] node{\texttt{)}} (q4)
    (b4) edge[above, ->] node{\texttt{[}} (q5)
    (b5) edge[above, ->] node{\texttt{]}} (q6)

    (b1) edge[above, ->] node{\texttt{\{}} (q7)
    (b2) edge[above, ->] node{\texttt{\}}} (q8)
    (b3) edge[above, ->] node{\texttt{:}} (q9)
    (b4) edge[above, ->] node{\texttt{,}} (q10)
    (b5) edge[above, ->] node{\texttt{;}} (q11)
    (b6) edge[above, ->] node{\texttt{+}} (q12)
    ;
\end{tikzpicture}
\caption{States \& transitions for recognising other single
letter tokens.}
\end{figure}

\subsection{The Director \& Builder}

Each drawing in \ref{sss:dfsadrawings} is directly represented
in code within the \texttt{LexerDirector} using methods provided
by the \texttt{LexerBuilder}.

\begin{lstlisting}[
caption={Code representation of \figref{comments} in the
\texttt{LexerDirector} class (lexer/LexerDirector.cpp).},
label=lst:commenttransitions]
// "/", "//", "/* ... */"
builder.addTransition(0, SLASH, 34)
    .setStateAsFinal(34, Token::Type::SLASH)
    .addTransition(34, SLASH, 35)
    .addComplementaryTransition(35, LINEFEED, 35)
    .setStateAsFinal(35, Token::Type::COMMENT)
    .addTransition(34, STAR, 36)
    .addComplementaryTransition(36, STAR, 36)
    .addTransition(36, STAR, 37)
    .addComplementaryTransition(37, SLASH, 36)
    .addTransition(37, SLASH, 38)
    .setStateAsFinal(38, Token::Type::COMMENT);
\end{lstlisting}

The \texttt{LexerBuilder} keeps track of these
transitions using less efficient data structures
such as hash maps (\texttt{std::unordered\_map})
and sets (\texttt{std::unordered\_set}).

Then the \texttt{build()} method processes the user defined
transitions and normalises everything into a single transition
table for use in a \texttt{DFSA}. Additionally, it also produces
two other artefacts. The first is called
\texttt{categoryIndexToChecker}. It is a hash map from the index
of a category to a lambda function which takes a character as
input and returns true or false.

The lambdas and category indices are registered by the user (see
\listref{checkerreg}). Additionally, the category indices
although they are integers, for readability they should be
defined as a (classical) enumeration.

\pagebreak

\begin{lstlisting}[ caption={Registration of the hexadecimal
($H$) category checker (lexer/LexerDirector.cpp).},
label=lst:checkerreg]
.addCategory(
    HEX,
    [](char c) -> bool {
        return ('0' <= c && c <= '9') ||
               ('A' <= c && c <= 'F') ||
               ('a' <= c && c <= 'f');
    }
)
\end{lstlisting}

The second artefact produced by the builder is also a hash map
whose keys are final state and values token types.

The transition table is then passed into a \texttt{DFSA}, and
the \texttt{DFSA}, along with the two other artefacts, is passed
into a \texttt{Lexer}.

\begin{lstlisting}[caption={Construction of a \texttt{Lexer}
(lexer/LexerBuilder.cpp).}]
// create dfsa
Dfsa dfsa(
    noOfStates,
    noOfCategories,
    transitionTable,
    initialStateIndex,
    finalStateIndices
);

// create lexer
Lexer lexer(
    std::move(dfsa),
    std::move(categoryIndexToChecker),
    std::move(finalStateIndexToTokenType)
);
\end{lstlisting}

\pagebreak

\subsection{The Actual \texttt{Lexer}}

A \texttt{Lexer}'s core, as described in class, is the method
\texttt{simulateDFSA()}.

\begin{note}
    The \texttt{Lexer} class has a number of very important
    auxiliary methods and behavioural changes. Specifically, the
    \texttt{updateLocationState()}, see \listref{updateloc}, is
    critical for providing adequate error messages both during
    lexing and later stages. This function is called every time
    a lexeme is consumed allowing the \texttt{Lexer} to keep
    track of where it is in the file (in terms of lines and
    columns).
\end{note}

\lstinputlisting[
firstline=110,
lastline=122,
caption={The \texttt{updateLocationState()} \texttt{Lexer}
method (lexer/Lexer.cpp).},
label=lst:updateloc
]{lexer/Lexer.cpp}

If an invalid/non-accepting state is reached the invalid lexeme
is consumed and the user is warned (see \listref{lexererrors}).
After such an encounter the \texttt{lexer}, is still in an
operational state. Hence, the \texttt{nextToken()} method can be
used again.

This is critical to provide users of the \texttt{PArL} compiler
with a list of as many errors as possible. Such behaviour is
desired because having to constantly run the \texttt{PArL}
compiler to see the next error, would be a bad user experience.

\begin{lstlisting}[caption={Error handling mechanism in the
\texttt{nextToken()} \texttt{Lexer} method (lexer/Lexer.cpp).},
label=lst:lexererrors]
if (state == INVALID_STATE) {
    mHasError = true;

    fmt::println(
        stderr,
        "lexical error at {}:{}:: unexpected "
        "lexeme '{}'",
        mLine,
        mColumn,
        lexeme
    );
} else {
    try {
        token = createToken(
            lexeme,
            mFinalStateToTokenType.at(state)
        );
    } catch (UndefinedBuiltin& error) {
        mHasError = true;

        fmt::println(
            stderr,
            "lexical error at {}:{}:: {}",
            mLine,
            mColumn,
            error.what()
        );
    }
}
\end{lstlisting}

\subsection{Hooking up the \texttt{Lexer} to the
\texttt{Runner}}\label{sss:runnerlexer}

The \texttt{Runner} class is the basic structure which connects
all the stages of the compiler together.

\lstinputlisting[
firstline=22,
lastline=28,
caption={The \texttt{Runner} constructor passing \texttt{mLexer}
into the \texttt{Parser} constructor (runner/Runner.cpp).}
]{runner/Runner.cpp}

Initially, the \texttt{Runner} passes a reference to the
\texttt{Lexer} into the \texttt{Parser}. This allows the
\texttt{Parser} to request tokens on demand improving the
overall performance of the compiler. Additionally, this has the
benefit of allowing the parsing of larger and multiple files.
This is because, the \texttt{Parser} is no longer limited by the
amount of usable memory, as it never needs to load whole files.

\begin{todo}
The groundwork for such an optimisation has been laid out
however, it is not being used. This is because the mechanism for
file reading requires changing.
\end{todo}
