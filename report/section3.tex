\section{Semantic Analysis}

\subsection{Phases \& Design}

As described in \ref{eyecandy} the Semantic Analysis phase
should be split into a number of sub-phases specifically: symbol
resolution, de-sugaring/type inference and type checking.

However, these steps can be combined together into a single step
phase. There a benefits and downsides to both approaches. The
main benefit of using the first approach is a reduction in
algorithmic complexity. The logic can be separated into
different phases with ease and information from one phase can be
propagated to another. The downside of this approach is that it
actually adds more code for maintenance, since each individual
sub-phase will probably need to be implemented as its own
visitor. The other down side is error management. If an error
occurs in a particular phase since said phase is completely
disjoint from the phases after it a mechanism for propagation
needs to be devised which again further increases complexity. Of
course by the very nature of this argument the monolithic
approach does not suffer for the issues that the sub-phases
approach has. However, it significantly increases complexity
since all sub-phases are being done in a single phase. The other
more glaring issue is the fact that it is much more difficult to
resolve symbols before-hand.

You would want to do so to allow for the location of function
declarations in code to not effect resolution, that is, a
function can be called before it is referenced.

Unfortunately, due to the fact that compiler development was an
organic process and not too much time was spend on deliberation
the semantic analysis phase became monolithic, and hence it
suffers from the issues described above. Of course, this means
location agnostic function declaration are currently \emph{not}
supported by the compiler.

\subsection{The Environment Data Structure}

The most important part of
