\section{Semantic Analysis}

\subsection{Phases \& Design}

As described in \ref{eyecandy} the Semantic Analysis phase
should be split into a number of sub-phases specifically: symbol
resolution, de-sugaring/type inference and type checking.

However, these steps can be combined together into a single step
phase. There a benefits and downsides to both approaches. The
main benefit of using the first approach is a reduction in
algorithmic complexity. The logic can be separated into
different phases with ease and information from one phase can be
propagated to another. The downside of this approach is that it
actually adds more code for maintenance, since each individual
sub-phase will probably need to be implemented as its own
visitor. The other down side is error management. If an error
occurs in a particular phase since said phase is completely
disjoint from the phases after it a mechanism for propagation
needs to be devised which again further increases complexity. Of
course by the very nature of this argument the monolithic
approach does not suffer for the issues that the sub-phases
approach has. However, it significantly increases complexity
since all sub-phases are being done in a single phase. The other
more glaring issue is the fact that it is much more difficult to
resolve symbols before-hand.

You would want to do so to allow for the location of function
declarations in code to not effect resolution, that is, a
function can be called before it is referenced.

Unfortunately, due to the fact that compiler development was an
organic process and not too much time was spend on deliberation
the semantic analysis phase became monolithic, and hence it
suffers from the issues described above. Of course, this means
location agnostic function declaration are currently \emph{not}
supported by the compiler.

\subsection{The Environment Tree}

Some terminology is required to properly describe the number of
structures which will be used in this section. The following
terms are critical and need to be differentiated properly:

\begin{itemize}
    \item \texttt{SymbolTable}
    \item \texttt{SymbolTableStack}
    \item \texttt{Environment}
    \item \texttt{EnvStack}
    \item \texttt{RefStack}
\end{itemize}

During semantic analysis there is a need for specific data
structures which facilitate scoping rules, type checking, etc.

The most basic data structure which achieves this, is a single
\texttt{SymbolTable}. A symbol table in its simplest form is a
wrapper around a hash map, whose keys are identifiers and values
are \texttt{Symbol}s or any structure which is capable of
storing variable and function signatures.

This approach is quite limiting since it restricts developers
and user of the language to a single global scope. Therefore,
this is structure is not a sufficient solution for most
toy-languages let alone production ready languages.

A better solution is using a stack of symbol tables. This allows
symbol tables to shadow each other allowing for the reuse of
identifiers. Additionally, this further opens up the possibility
of implementing modules at the language-level. The likelihood
that names will be reused across different modules is quite high
and being able to scope modules so they do not interfere with
global scope and each other is necessary for any sufficiently
large project.

\begin{figure}[H]
\centering
\begin{mdframed}[backgroundcolor=UMPaleRed]
\includegraphics[width=\linewidth]{scopedcode.pdf}
\end{mdframed}
\caption{A \texttt{PArL} function with annotated scopes}
\label{fig:scopeannotatedcode}
\end{figure}

The basic premise of using a symbol table stack is described in
\algref{basicsymboltablestack} and \figref{graphicaldecpiction}.
A new symbol table, also referred to as a scope, is pushed onto
a stack only for specific types of AST nodes. The node is then
processed, were ``processing'' often times means considering all
the sub-children of the node, and when processing terminates the
scope is popped. In the context of a purely stack based
implementation this implies that, scopes are only temporary that
is they are lost after being popped.

\begin{algorithm}[H]
\KwData{$N$ the AST node, $S$ the symbol table stack}

\Begin{
    \If{$N$ opens a scope}{
        PushScope($S$)\;
    }
    ProcessNode($N$)\;
    \If{$N$ opens a scope}{
        PopScope($S$)\;
    }
}

\caption{Basic description of \texttt{SymbolTableStack} usage}
\label{alg:basicsymboltablestack}
\end{algorithm}

\begin{figure}[H]
\centering
\begin{mdframed}[backgroundcolor=UMPaleRed]
\includegraphics[width=\linewidth]{symboltablestack.pdf}
\end{mdframed}
\caption{Behaviour of the \texttt{SymbolTableStack} when
considering the code in \figref{scopeannotatedcode}}
\label{fig:graphicaldecpiction}
\end{figure}

\label{sss:memoryadvantage}The main advantage of using this
approach is that the worst case memory usage is $O(DI)$ where
$D$ is the length of a longest chain of open scopes and $I$ is
the size of largest number of variable declarations within a
scope.

However, due to the temporary nature of this approach it does
not lend itself well to a multi-phase approach. This is because
at each phase \texttt{SymbolTable}s have to be regenerated
increase the complexity of each individual phase.

Instead a different approach shall be used. This approach uses
an
\href{https://craftinginterpreters.com/statements-and-state.html#nesting-and-shadowing}{Environment
Tree} and it has also been inspired by
\href{https://craftinginterpreters.com/}{Crafting Interpreters}.

The notion of an \texttt{Environment} is essentially, the same
as a \texttt{SymbolTable}. The only significant change is the
addition of vector of unique pointers to other environments, see
\listref{envclass}.

\begin{lstlisting}[escapechar=!,caption={The
\texttt{Environment} class with \texttt{mChildren} highlighted
(backend/Environment.hpp)}, label=lst:envclass]
class Environment {
   public:
    enum class Type {
        GLOBAL,
        IF,
        ELSE,
        FOR,
        WHILE,
        FUNCTION,
        BLOCK
    };

    void addSymbol(
        std::string const& identifier,
        Symbol const& Symbol
    );
    [[nodiscard]] std::optional<Symbol> findSymbol(
        std::string const& identifier
    ) const;
    Symbol& getSymbolAsRef(std::string const& identifier);

    [[nodiscard]] Environment* getEnclosing() const;
    void setEnclosing(Environment* enclosing);

    [[nodiscard]] Type getType() const;
    void setType(Type type);

    [[nodiscard]] std::optional<std::string> getName(
    ) const;
    void setName(const std::string& name);

    std::vector<std::unique_ptr<Environment>>& children();

    [[nodiscard]] bool isGlobal() const;

    [[nodiscard]] size_t getIdx() const;
    void incIdx();
    void incIdx(size_t inc);

    [[nodiscard]] size_t getSize() const;
    void setSize(size_t size);

   private:
    std::unordered_map<std::string, Symbol> mMap{};
    Type mType{Type::GLOBAL};
    std::optional<std::string> mName{};
    Environment* mEnclosing{nullptr};
    !\colorbox{UMPaleRed}{std::vector<std::unique_ptr<Environment>> mChildren\{\};}!
    size_t mSize{0};
    size_t mIdx{0};
};
\end{lstlisting}

The additional fields present in the \texttt{Environment} class
are there to cater for the needs of each of the phases.

Specifically,\label{sss:extrafields} \texttt{mType} and
\texttt{mName} are used during semantic analysis and
\texttt{mSize} and \texttt{mIdx} are used during code
generation.

Additionally, it is preferable if the interface with which the
phases interact with the \texttt{Environment} tree is identical
to that of a \texttt{SymbolTableStack}, that is the same
\texttt{push()} and \texttt{pop()} methods are used.

Now to facilitate this another two classes need to be defined
\texttt{EnvStack} and \texttt{RefStack}. The main difference
between \texttt{EnvStack} and \texttt{RefStack} is that the
purpose of \texttt{EnvStack} is to construct the
\texttt{Environment} tree whilst \texttt{RefStack} traverses the
environment tree. Due to this \texttt{RefStack} is only required
during the symbol resolution. However, in current implementation
since symbol resolution is combined with type checking, the
Environment tree is generated at the same time as it is being
type checked.

Additionally, in both an \texttt{EnvStack} and a
\texttt{RefStack}, the creation and traversal of the Environment
tree are managed via the \texttt{push()} and \texttt{pop()}
methods.

In an \texttt{EnvStack} \texttt{push()} works by creating a new
environment, setting the enclosing environment to the current
environment and changing the current environment to to the new
environment, see \listref{evnstackpush}. \texttt{pop()} makes
use of the \texttt{mEnclosing} field in the current environment
and just sets the current environment to the enclosing
environment essentially moving up an environment, see
\listref{popenvstack}.

\lstinputlisting[linerange={12-22}, caption={The
\texttt{pushEnv()} method in the \texttt{EnvStack} class
(analysis/EnvStack.cpp)}, label=lst:envstackpush
]{analysis/EnvStack.cpp}

\lstinputlisting[linerange={24-28}, caption={The
\texttt{popEnv()} method in the \texttt{EnvStack} class
(analysis/EnvStack.cpp)}, label=lst:envstackpop
]{analysis/EnvStack.cpp}

The implementations in \texttt{RefStack} are however a bit more
involved. This is because the program has to be able to perform
a step-wise left-first depth-first traversal. The best way
to do this is to make use of a stack and the algorithms
in \listref{refpush} and \listref{refpop}.

\lstinputlisting[linerange={7-17}, caption={The
\texttt{pushEnv()} method in the \texttt{RefStack} class
(ir\_gen/RefStack.cpp)}, label=lst:refpush
]{ir\_gen/RefStack.cpp}

\lstinputlisting[linerange={37-43}, caption={The
\texttt{popEnv()} method in the \texttt{RefStack} class
(ir\_gen/RefStack.cpp)}, label=lst:refpop
]{ir\_gen/RefStack.cpp}

See \figref{refstackdryrun}, for the initial segments of a
traversal. The important thing to note here is that the
correctness of the traversal entirely depends on the usage of
\texttt{push()} and \texttt{pop()}. If used incorrectly the
traversal might be incorrect or the program might crash. Of
course, this is only an implementation detail that is if used
correctly by the compiler developer no such issue should occur.

\begin{figure}[H]
\centering
\begin{mdframed}[backgroundcolor=UMPaleRed]
\includegraphics[width=\linewidth]{refstackdryrun.pdf}
\end{mdframed}
\caption{The initial steps of a traversal performed by
\texttt{RefStack}}
\label{fig:refstackdryrun}
\end{figure}

In the current implementation of \texttt{PArL} three types of
nodes control scopes, these are: \texttt{Block}s,
\texttt{ForStmt}s and \texttt{FunctionDecl}s. There is also a
slight particularity in that, \texttt{ForStmt}s and
\texttt{FunctionDecl}s as per the grammar actually, casue two
scopes to open. This really depends on whether shadowing of,
generally the iterator in a for-loop and the function parameters
in the language should be allowed or not. For simplicities sake
in \texttt{PArL} a second scope is being opened. However there
is no particular consensus, for example in \CC{} shadowing of
function parameters or declarations in for-loops is not allowed
since they create a single scope, but Rust allows shadowing for
the same variable in the same scope without any problems.

\subsection{Semantic Analysis}

Having the necessary data structures in place the discussion
will now revolve around Semantic Analysis i.e. the process of
making sure the meaning of the program is correct.

There are four main areas of interest which shall be discussed:
the \ul{symbol type}, \ul{symbol registration/search}, \ul{type
checking} and a unique subset of type checking, \ul{return value
type verification}.

\subsection{The \texttt{Symbol} Type}

Within \texttt{PArL} functions are special they are not treated
as values i.e. they they cannot be assigned to variables, they
cannot be moved around etc. This means that the \texttt{Symbol}
type is actually larger than the \texttt{Primitve} type
referenced in \ref{sss:primitive}. It has to support both
variables and functions, hence the \texttt{Symbol} is truly the
union of two different types, \texttt{VariableSymbol} and
\texttt{FunctionSymbol}.

\begin{lstlisting}[caption={Definitions of
\texttt{VariableSymbol} and \texttt{FunctionSymbol}
(backend/Symbol.hpp)},label=lst:symbols]
struct VariableSymbol {
    size_t idx{0};
    core::Primitive type;

    ...
};

struct FunctionSymbol {
    size_t labelPos{0};
    size_t arity{0};
    std::vector<core::Primitive> paramTypes;
    core::Primitive returnType;

    ...
};
\end{lstlisting}

Again the additional fields present, see \listref{symbols},
within said symbols helps reduce complexity further down the
pipeline especially in code generation.

Additionally, it is important to note that only primitives are
comparable. Of course being able to compare the full symbols or
adding said logic in the \texttt{Symbol} class goes against
separation of behaviour and state.

\subsubsection{Symbol Registration \& Search}

A Symbol in \texttt{PArL} is an attribute associated with an
identifier (a string). One of the main jobs of semantic analysis
is to solidify these associations from the AST into the current
Environment. In \texttt{PArL} the only nodes which are meant to
cause a registration are the Variable Declaration node and the
Function Declaration node.

The only important considerations for registration are that a
identifier is never registered twice with in the same scope and
that function identifiers are globally unique, that is they
cannot be shadowed.

\lstinputlisting[ linerange={926-963}, caption={The
\texttt{visit(FormalParam *)} method in the
\texttt{AnalysisVisitor} class (analysis/AnalysisVisitor.cpp)},
label=lst:internaltypesys ]{analysis/AnalysisVisitor.cpp}

Searching happens when a identifier in a none-declaration node
is met. The procedure for searching basically requires querying
the current environment for the symbol associated with the found
identifier. If nothing is found climb to the enclosing state and
repeat the process. If the root node or \emph{terminating node}
is reached and no associated symbol is found that means that the
identifier is undefined.

Note that the specification of a \texttt{terminating node} (a
\texttt{Environment*}) is important. This is because it allows
the analyser to restrict access to outer scopes when necessary.
The main example of this is function declarations. In
\texttt{PArL} functions are (for the most) part pure, that is
they do not produce side-effect at a language level. To enforce
this, functions are not allowed to access any state outside of
their own scope. The only exception to this fact is the usage of
built-ins like \texttt{__write} which directly effect the
simulator.

This is where one of the extra fields mentioned in
\ref{sss:extrafields}, specifically \texttt{mType} is used
because it allows scopes to be of different types and this is
necessary to find function scopes. For example in
\listref{visitvariable} \texttt{findEnclosingEnv()} is being
used to find the closest enclosing function scope and if no such
scope is found i.e. the optional is empty. Then searching either
terminates on the stopping scope which is either global or the
enclosing function scope.

\begin{lstlisting}[caption={The \texttt{visit(Variable *)}
method in the \texttt{AnalysisVisitor} class
(analysis/AnalysisVisitor.cpp)},label=lst:visitvariable]
void AnalysisVisitor::visit(core::Variable *expr) {
    std::optional<Environment *> stoppingEnv =
        findEnclosingEnv(Environment::Type::FUNCTION);

    std::optional<Symbol> symbol{findSymbol(
        expr->identifier,
        stoppingEnv.has_value() ? *stoppingEnv
                                : mEnvStack.getGlobal()
    )};

    if (!symbol.has_value()) {
        error(
            expr->position,
            "{} is undefined",
            expr->identifier
        );
    }

    ...
\end{lstlisting}

\subsection{Type Checking}

This is the more tedious aspect of semantic. Each individual
case where the types affect the behaviour of program or are
simply not allowed have to be considered. For this the
\texttt{AnalysisVisitor} has the field \texttt{mReturn} which is
essentially used as the return of a visit, see
\listref{binaryreturn}.

\begin{lstlisting}[caption={A part of the \texttt{visit(Binary
*)} method in the \texttt{AnalysisVisitor} class
(analysis/AnalysisVisitor.cpp)},label=lst:binaryreturn]
void AnalysisVisitor::visit(core::Binary *expr) {
    expr->left->accept(this);
    auto leftType{mReturn};

    expr->right->accept(this);
    auto rightType{mReturn};

    ...
\end{lstlisting}

In terms of design choice a strict approach was chosen, that is
no implicit casting takes place. Operations are not capable of
operating with two distinct types, for example \texttt{5 / 2.0}
is not allowed. Instead operations which should support multiple
types have been overloaded (later on in code generation). For
example, \texttt{__print 5 / 2} outputs 2, \texttt{__print 5 /
2.0} is a type error and \texttt{__print 5.0 / 2.0} outputs 2.5.

\begin{lstlisting}[caption={Another part of the
\texttt{visit(Binary *)} method in the \texttt{AnalysisVisitor}
class (analysis/AnalysisVisitor.cpp)},label=lst:typechecking2]
        ...

        case core::Operation::ADD:
        case core::Operation::SUB:
            if (leftType.is<core::Array>()) {
                error(
                    expr->position,
                    "operator {} is not defined on array "
                    "types",
                    core::operationToString(expr->op)
                );
            }

            if (leftType != rightType) {
                error(
                    expr->position,
                    "operator {} expects both operands to "
                    "be of same type",
                    core::operationToString(expr->op)
                );
            }

            mReturn = leftType;
            break;

            ...
\end{lstlisting}

Of course the remainder of the \texttt{AnalysisVisitor}
is essentially, type checking in the form of described
in \listref{typechecking2}.

\subsubsection{Return Validation}

\begin{lstlisting}[caption={Checking whether return statement is
inside a function declaration in the \texttt{visit(ReturnStmt
*)} method in the \texttt{AnalysisVisitor} class
(analysis/AnalysisVisitor.cpp)},label=lst:returninsidefunc]
    ...

    std::optional<Environment *> optEnv =
        findEnclosingEnv(Environment::Type::FUNCTION);

    if (!optEnv.has_value()) {
        error(
            stmt->position,
            "return statement must be within a "
            "function block"
        );
    }

    ...
\end{lstlisting}

The only non-trivial check is that of a return statement. First
of all return statements must be enclosed within functions,
otherwise it is a semantic error, see
\listref{returninsidefunc}.

\begin{lstlisting}[caption={Checking whether the return
expression has the same type as the function return type in the
\texttt{visit(ReturnStmt *)} method in the
\texttt{AnalysisVisitor} class
(analysis/AnalysisVisitor.cpp)},label=lst:returntype]
    ...

    Environment *env = *optEnv;

    std::string enclosingFunc = env->getName().value();

    auto funcSymbol = env->getEnclosing()
                          ->findSymbol(enclosingFunc)
                          ->as<FunctionSymbol>();

    if (exprType != funcSymbol.returnType) {
        error(
            stmt->position,
            "incorrect return type in function {}",
            enclosingFunc
        );
    }

    ...
\end{lstlisting}

Additionally, they must have the same return type as the
enclosing function, see \listref{returntype}, and finally and
most importantly, \textbf{a function must return in all possible
branches}. Additionally, the implementation \textbf{should be
capable of recognising unreachable areas of code}.

\begin{marker}
\label{sss:returnissue} Not stopping when an unreachable code
segment is met could result in an invalid assessment of whether
a function returns from all branches or not.
\end{marker}

During the first iteration of the compiler, the mechanism for
checking the return type was embedded directly into the semantic
analysis. However, their is slight problem with this approach.
If an unreachable segment of code is met to avoid the
\hyperref[sss:returnissue]{issue} mentioned above the remainder
of the unreachable code will not be type checked. Of course this
is a problem since such behaviour means the compiler can ignore
semantically incorrect code and still produce VM instructions.

To solve this issue a separate visitor \texttt{ReturnVisitor}
was created to handle checking that functions returns from all
possible branches. The separation facilitates checking after
type checking is complete circumventing any of the discussed
issues.

The mechanism which \texttt{ReturnVisitor} uses to establish
whether a function returns or not is using a class member
\texttt{mBranchReturns}.

This member is a boolean and it is explicitly set to true only
by a return statement. Function declarations reset the
\texttt{mBranchReturns} to false and other statements which do
not exhibit branching do not modify the value of
\texttt{mBranchReturns}. This means that the only statements
which effect \texttt{mBranchRetruns} are if-else statements,
for-loops and while-loops, since they branch.

For-loops and while-loops reset \texttt{mBranchReturns} to false
after their block of statements. This is because entry into the
body of the loop (both while and for) is conditional and hence a
program might never enter into the body of the loop. Hence, it
is not enough to return from the body of the loop.

\begin{lstlisting}[caption={The \texttt{visit(IfStmt *)} method
in the \texttt{ReturnVisitor} class
(analysis/ReturnVisitor.cpp)},label=lst:iftypecheck]
void ReturnVisitor::visit(core::IfStmt *stmt) {
    stmt->thenBlock->accept(this);
    bool thenBranch = mBranchReturns;

    bool elseBranch = false;

    if (stmt->elseBlock) {
        stmt->elseBlock->accept(this);
        elseBranch = mBranchReturns;
    }

    mBranchReturns = thenBranch && elseBranch;
}
\end{lstlisting}

If-else statements are different in that if the `then' block
returns and the `else' block returns then the whole if-else
statement returns otherwise the if-else statement does not
return. This is essentially a logical and between the results of
the visitor for the `then' and `else' blocks, see
\listref{iftypecheck}.

\begin{lstlisting}[escapechar=!, caption={The \texttt{visit(Block *)} method
in the \texttt{ReturnVisitor} class
(analysis/ReturnVisitor.cpp)},label=lst:dropinblock]
void ReturnVisitor::visit(core::Block *block) {
    size_t i = 0;

    for (; i < block->stmts.size(); i++) {
        block->stmts[i]->accept(this);

        if (mBranchReturns) {
            i++;

            break;
        }
    }

    if (i < block->stmts.size()) {
        core::Position start = block->stmts[i]->position;

        warning(
            start,
            "statements starting from line {} upto line {} "
            "are unreachable",
            start.row(),
            block->position.row()
        );
    }

    !\colorbox{UMPaleRed}{
    for (; i < block->stmts.size(); i++) \{\ block->stmts.pop_back();\ \}}!
}
\end{lstlisting}

Finally, in the block statements if one of the statements inside
manages to successfully return, the remaining statements will
never be reached and hence the user can be notified and the
statements can be dropped, see \listref{dropinblock}.

The main


To do this

This minor change removes the need to throw away Environments
and they can be passed on from one phase to another.

\begin{figure}[H]
\centering
\begin{mdframed}[backgroundcolor=UMPaleRed]
\includegraphics[width=\linewidth]{scopedcode2.pdf}
\end{mdframed}
\caption{Construction of an \texttt{Environment} tree}
\label{fig:scopedcode2}
\end{figure}

The Environment tree is a data structure

\begin{itemize}
    \item Describe Symbol Table.
    \item Describe each element of the symbol table.
    \item Describe the tree approach.
    \item attribute the tree approach to Robert Nystrom as well.
    \item describe the main advantage/disadvantage of using a environment
        approach.
    \item describe the main advantage/disadvantage of using a symbol stack
        approach.
\end{itemize}
